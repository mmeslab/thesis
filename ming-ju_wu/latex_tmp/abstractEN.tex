\begin{abstractEN}

This dissertation proposes a novel operating system (OS) state pausing methodology that can be used to enable the co-simulation of virtually simulated storage device and physical computer system in the discrete-time domain. 

In storage device emulation, the emulated virtual storage device appears to the OS as a real storage device. The service timings of the emulated storage device are determined by a disk model which simulates the characteristics and performance of the target storage device. For the conventional storage device emulators, because the OS is running continuously in the real-time domain, the amount of time that the disk emulator can spend on processing each I/O request is limited by the servicing time of the corresponding I/O request. If the processing of an I/O request takes longer than the servicing time of the I/O request, then the OS would perceive a storage device that has a slower performance than the intended target. This real-time timing constraint can make emulating high-speed storage devices a challenge for conventional storage device emulators.

By utilizing the proposed OS state pausing co-simulation methodology for emulating storage devices, the timing constraints faced by the conventional storage device emulators can be avoided. By pausing the state of the OS whenever the storage device emulator is busy, the emulator can spend as much time as it needs for processing each I/O request without affecting the performance of the emulated storage device as perceived by the OS. In addition, because the OS and application programs are executing directly on the physical computer hardware, the performance characteristics of the real system can be taken into account during the co-simulation process.

In this dissertation, two co-simulation environments are constructed using the proposed OS state pausing methodology. (1) The first environment is a full-system co-simulator that runs the storage device emulator directly on the OS that the emulated storage device is plugged into. The main focus of this co-simulation environment is to allow realistic workloads and system-level metrics to be used for evaluating storage subsystem designs. With the experimental configuration, the proposed co-simulation environment has achieved simulation speed of up to 45 times faster than real-time. (2) The second co-simulation environment is focused on predicting the overall system performance of a particular computer system when different storage devices are available. To minimize the disturbances that the storage device emulator will have to the target system, and therefore improve the prediction accuracy, the work of storage device emulation is offloaded to an external computer. Experimental results show that the full-system performance benchmark results predicted using the proposed storage device emulator are within 2\% differences compared to the results from the reference system.

\end{abstractEN}