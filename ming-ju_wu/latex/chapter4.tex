\chapter{Previous Work}
\label{ch:4}

The key concept of the proposed co-simulation methodology is to control the execution state of the OS and make the OS to observe a virtual system time that is not hardwired to the real world clock. In the work of Gupta~\cite{Gupta:2006}, they proposed a technique called \textit{time dilation} to make the OS to observe a passage of time that is a constant time slower than the real-time clock. From the OS’s perspective, the physical resources in the external world will appear to be faster than their original speeds. For example, if the passage of time observed by the OS is slowed down by 10 times, then data arriving from a network interface at a physical rate of 1Gbps would appear to the OS as arriving at 10Gbps. They have demonstrated that \textit{time dilation} is an effective method for emulating network interfaces of different speeds to the OS for end-to-end system behavior experimentations. 

Although the proposed storage device emulator also makes the OS to observe a passage of system time that is different than the real-time clock, the goal and mechanisms are different from those of \textit{time dilation}:

\begin{itemize}
	\item  In \textit{time dilation}, the passage of system time is only slowed down by a constant factor but will never be stopped. In other words, the OS is always executing continuously. In contrast, in our proposed OS state pausing approach, the execution of the OS will be paused as necessary whenever the storage device emulator is busy processing the I/O requests. The goal of \textit{time dilation} is to trick the OS into believing that the external resources are faster than what they actually are. On the other hand, the goal of OS state pausing is to temporarily freeze the execution of the OS so that the storage device emulator can spend unlimited amount of time on processing the I/O requests.
	
	\item The slowing down of the passage of time in \textit{time dilation} is achieved by reducing the frequency that the timer interrupts are delivered to the OS and also by appropriately scaling down the hardware time counter that is used by the OS. For example, to slow down the passage of time observed by the OS by 10 times, the timer interrupt frequency is reduced by 10 times and the hardware time counter is also scaled down by 10 times. In comparison, pausing the state of the OS requires stopping the hardware time counter and preventing the CPU from doing any work for the OS.
	
	\item The design of the proposed storage device emulator can be extended with \textit{time dilation}. In the proposed emulator design, the OS can either be in the paused state or in the running state. When in the running state, it is possible to apply \textit{time dilation} so that the OS will observe that it is running on a CPU which is faster than the CPU of the target system. However, in order to keep our focus, we do not apply \textit{time dilation} to the proposed storage device emulator in this article.
\end{itemize}

Proper evaluation of storage subsystems is not an easy task. Traeger et al.~\cite{Traeger:2008} have proposed a set of guidelines for proper storage subsystem performance evaluations.
% TODO: Summarize the guide lines here.
Using the guidelines as the evaluation criteria, they have surveyed 415 file system and storage benchmarks from a selection of conference papers between the year 1999 and 2007. The selected benchmarks are all conducted on real machine environments. They have found that most benchmarks are flawed and could give the readers wrong impressions on the real performance of the storage subsystems under test and even lead to incorrect design conclusions. From the guidelines, it can be clearly seen that it is important that the system-level behaviors are taken into account when performing storage subsystem evaluation.

In the work of Ganger and Patt~\cite{Ganger:1998}, they have constructed abstract system-level model that contains necessary host system component modules with appropriate details for generating workloads for benchmarking storage subsystems. System-level traces, instead of I/O traces, are recorded on real system and played back using the abstract system-level model for generating workloads to drive the storage subsystem module.

Instead of using purely abstract models for system-level simulation, Thekkath~\cite{Thekkath:1994} incorporated real file system code into the simulation system. In his approach, I/O traces are gathered on real systems and are used to drive the real file system code in the simulation process for generating I/O requests.

An alternative to using abstract system-level models for performing storage subsystem evaluations is to use complete machine simulation. The SimOS environment~\cite{Rosenblum:1995} \cite{Witchel:1996} \cite{Rosenblum:1997} is a complete machine simulator which can simulate the hardware of an SGI machine in enough detail to run the Irix OS. SimOS supports switching among a number of hardware component models which are different in terms of modeling detail and accuracy. The less detailed models can run faster and can be used to boot and position the system into a state at which interesting workloads will begin. The simulation can then switch to use a more accurate model for detailed profiling. Griffin et al. [2000] has integrated a MEMS-based storage device simulator into the SimOS environment for studying the performance and characteristics of MEMS-based storage devices. Although complete machine simulation does allow the actual OS and real applications to be run on the simulated machine hardware, it can run much slower than the real system. Furthermore, building and maintaining accurate complete machine simulation models could take substantial amount effort.

Timing-accurate storage emulation has been explored in detail by \cite{Griffin:2002}. Memulator allows researchers to explore with nonexistent storage devices and take end-to-end measurements in full-system contexts. Memulator can use either the memory of the local machine or a networked machine as the backing storage for the emulated disk. Their work gives support that storage device emulation is an effective method for studying the overall system performance with simulated virtual storage devices. In their remote emulation setup, the emulator is executed on a standalone computer and connected to the target system over the SCSI interface. The DiskSim simulator~\cite{Bucy:2008} is used for simulating the service times of the Seagate Cheetah X15 hard disk drive. Experimental results show that the benchmark results measured with the emulated storage device are close to the results measured with the real storage device.

In the work of Maghraoui et al.~\cite{ElMaghraoui:2010}, a local emulation type storage device emulator is developed to emulate flash based SSD for the AIX OS. The emulator is implemented as a dynamic loadable kernel module and appears to the OS as a disk device. A chunk of system memory is pined for use as the backing storage, and high resolution nanosecond granularity timers are used for simulating the delays of the I/O operations. Lee et al.~\cite{Lee:2012} have proposed another local emulation type SSD emulator. The target platform is the Linux OS and it uses a user mode SSD simulation engine for calculating the latencies for the I/O requests. The backing storage used can be either the system memory or some other external DRAM-based storage devices.

David~\cite{Agrawal:2012} is another local emulation type storage device emulator which is mostly concerned about the required backing storage size for emulating large storage devices. The main idea of David is to reduce the data that needs to be saved to the backing storage so that larger storage devices can be emulated using less storage space. The key idea is that for certain types of benchmark workloads, the actual content in the file system files can be omitted and only file system metadata needs to be persisted onto the backing storage. By not persisting the actual content in the files, David is able to reduce the storage size requirements by orders of magnitude. David has been demonstrated to work with the ext3 and the btrfs file systems, and in theory, it can be extended to work with other types of file systems.

VSSIM~\cite{Yoo:2013} is a virtual machine (VM) based SSD simulator which makes emulated SSD devices available to the VM. The emulated SSD device appears to the VM as a storage device connected to the IDE interface. VSSIM runs in real-time and allows the user to measure both the host performance and the SSD behavior under various design choices. The design of VSSIM can conceptually be classified as similar to the remote emulation design. The difference is that the target machine that the emulated storage device is ``connected'' to is a VM, and therefore VSSIM does not need to emulate the actual electrical signals of the IDE interface.

Finally, Canon et al.~\cite{Canon:1980} extended the standard virtual machine (VM) environment to include virtual time emulation. In contrast to the conventional VM environment, in which the real time of day (RTOD) clock is observed by the VM, the modified VM environment makes the VM to observe a virtual time of day (VTOD) clock. The VTOD clock progresses as program instructions are executed on the virtual CPU of the VM. In the modified VM environment, the system performance is evaluated against the VTOD clock. The VM environment proposed by Canon et al. can be viewed as analogous to the SimOS environment configured with the direct execution model, in which the machine instructions are executed on the native CPU whenever possible and that the internal architecture of the CPU is not modeled.
